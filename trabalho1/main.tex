% Explanação ====================================
% Estas são algumas falas de um trabalho da escola
% Eu estou usando o LaTeX, um sistema de preparação de documentos
% As partes do texto que estiverem com um % significa que são anotações, ou em termos de programação "um comentário"
% ===============================================

\documentclass{article}  % Explanação rápida: Isso define o tipo do documento

% Pacotes =======================================
% Pacotes são coisas que você usa para melhorar ser documento.
% Pacotes que eu estou usando:
% Anyfontsize: Para aumentar o tamanho da fonte
% Babel: Para deixar o documento em português
% Geometry: Pra ajustar o documento
% Parskip: Pra controlar os parágrafos
% ===============================================

\usepackage{anyfontsize}
\usepackage[portuguese]{babel}  % Explanação rápida: Esta configuração traduz o documento para português
\usepackage[left=3cm,top=3cm,right=2cm,bottom=2cm,a4paper]{geometry}  % Explanação rápida: essas configurações ajustam o tamanho do documento (falando especificamente as definições, a margem esquerda e a margem superior têm 3 centímetros e a margem direita e a margem inferior têm 2 centímetros)
\usepackage[indent=0cm,skip=0.1cm]{parskip}  % Explanação rápida: Essa configuração controla os parágrafos do documento

% Informações iniciais ==========================
% Nas linhas 30 a 33 temos algumas informações
% \title{Falas}: Isso significa que o título do documento é "Falas"
% \author{David, Wenderson, Isaac, Otávio e Marcos}: Isso são os nomes dos autores do documento
% \date{5 de novembro de 2023}: A data que o documento foi feito
% ===============================================

\title{Falas}
\author{David, Wenderson, Isaac, Otávio e Marcos}
\date{5 de novembro de 2023}

\begin{document}  % Explanação rápida: Isso começa o documento
\fontsize{12pt}{18pt}\selectfont  % Explanação rápida: Isso aumenta o tamanho da fonte

\maketitle  % Explanação rápida: Isso coloca um título bem formatado no documento com base nas informações dadas no \title, \author e \date 

% Uma coisa importante ==========================
% Existêm dois arquivos importantes, um deles é para as falas e outro é para o pedido especial, eles estão em uma pasta chamada secoes
% ===============================================

\section{Falas:}
\subsection{Fala 1:}
\textit{(Provavelmente um "Olá" ou alguma outra saudação. Como por exemplo "Olá a todos, meu nome é... do sexto ano e hoje vamos apresentar a imunização da população durante a COVID-19")}  % Esse comando coloca o conteúdo do arquivo falas.tex, que está na pasta seções nesse arquivo
\section{Pedido especial:}
Por favor, caso queira ver como as falas foram feitas ou quer mudar algo, acesse o nosso repositório do GitHub em \textbf{https://github.com/DavdTheItGuy/Projetos-da-Escola/blob/main/trabalho1/Falas.pdf}.
\end{document}  % Explanação rápida: Isso termina o documento
